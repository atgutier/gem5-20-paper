\subsection[Arm Improvements]{Arm Improvements}
\label{sec:arm}

\subsubsection[Armv8 Support]{Armv8 Support\footnote{by Giacomo Gabrielli, Andreas Sandberg, and Giacomo Travaglini}}

The Armv8-A architecture introduced two different architectural states:
AArch32, supporting the A32 and T32 instruction sets (backward-compatible with
Armv7-A and Thumb instruction sets, respectively), and AArch64, a new
state offering support for 64-bit addressing via the A64 instruction set. Currently, gem5
supports all of the above instruction sets and the interworking
between them.
On top of the user-level features, several important system-level extensions, e.g. the
security (aka TrustZone\textregistered~\cite{ArmTustZone}) and virtualization extensions~\cite{ArmARM}, have been contributed opening up new avenues for architectural and microarchitectural research.

While Armv8-A was a major iteration of the architecture, there have been
several smaller iterations introduced by Arm with a yearly cadence, and various
contributors have implemented some of the main features from those extensions,
up to Armv8.3-A.

\subsubsection[Support for the Arm Scalable Vector Extension (SVE)]{Support for the Arm Scalable Vector Extension (SVE)\footnote{by Giacomo Gabrielli, Javier Setoain, and Giacomo Travaglini}}

In 2016, Arm introduced their Scalable Vector Extension (SVE)~\cite{ArmARM}, a
novel approach to vector instruction sets. Instead of having fixed-size vector
registers, SVE operates on registers that can be anywhere between 128 to 2048
bit long (in 128-bit increments). SVE code is arranged in a way that is agnostic to the
underlying vector length (Vector Length Agnostic Programming), and a single SVE
instruction will perform its operation on as many elements as the vector
register can fit, depending on its length. On top of the 32 variable-length
vector registers, SVE also adds 16 variable length predicate registers for
predicated execution. These registers store one bit per byte (the minimum
element size) in the vector register, and can be used to select specific
elements in the vector for operation~\cite{white-paper-on-SVE-and-VLA-programming}.

To support SVE, gem5 implements register storage and register access
as two separated classes, a container and an interface, decoupling one from the
other. The vector registers can be of any arbitrary size and be accessed as
vectors of elements of any particular type, depending on the operand types of
each instruction. This not only facilitates handling variable size registers,
it also abstracts the nuances of handling predicate registers, where the stored
values have to be grouped and interpreted differently depending on the operand
type.

This design provides enough flexibility to support any vector instruction sets
with arbitrarily large vector registers.

\subsubsection[Trusted Firmware Support]{Trusted Firmware Support\footnote{by Adrian Herrera}}

Trusted Firmware (TF-A) is Arm's reference implementation of Secure World software for A-profile architectures.
It enables Secure Boot flow models, and provides implementations for the Secure Monitor executing at Exception Level 3 (EL3) as well as for several Arm low-level software interface standards, including System Control and Management Interface (SCMI) driver for accessing System Control Processors (SCP), Power State Coordination Interface (PSCI) library support for power management, and Secure Monitor Call (SMC) handling.

TF-A is supported on multiple Arm Development Platforms (APDs), each of them characterized by its set of hardware components and their location in the memory map (e.g., Juno ADP and the Fixed Virtual Platforms (FVP) ADP family).
However, the Arm reference platforms in gem5 are part of the \verb|VExpress_GEM5_Base| family.
These are loosely based on a Versatile\texttrademark Express RS1 platform with a slightly modified memory map. TF-A implementations are provided for both Juno and FVPs, however not for \verb|VExpress_GEM5_Base|.

Towards unifying Arm's platform landscape, we now provide a \verb|VExpress_GEM5_Foundation| platform as part of gem5's \verb|VExpress_GEM5_Base| family.
This is based on and compatible with FVP Foundation, meaning all Foundation software may run unmodified in gem5, including but not limited to TF-A.
This allows for simulating boot flows based on UEFI implementations (U-boot, EDK II), and brings us a step closer to support for all UEFI compatible Operating Systems in gem5.
